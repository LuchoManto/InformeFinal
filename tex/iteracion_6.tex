\chapter{Iteracion 6: Implementacion de un sistema gestionador para la placa de instrumentacion} % (fold)
\label{cha:iteracion_6}

\section{Introduccion} % (fold)
\label{sec:introduccion}

El sistema de instrumentacion realiza mediciones para luego enviarselas a otro sistema que las gestiona o realiza acciones segun las mismas. En esta iteracion, realizamos una implementacion de un sistema que actua entre el dispositivo que maneja el usuario y la placa de instrumentacion. De esta forma obtuvimos un intermediario que interpreta acciones del usuario y las transforma en comandos para la placa de instrumentacion.

% section introduccion (end)

\section{Requerimientos de la iteracion} % (fold)
\label{sec:requerimientos_de_la_iteracion}

\begin{itemize}
\item El sistema deberia estar implementado en una placa de desarrollo de manera que se pueda conectar a la placa de instrumentacion via comunicacion serial
\item Deberia estar en el mismo lugar fisico que la placa de alimentacion.
\item Deberia implementar un servidor web con una interfaz grafica de usuario. Esta interfaz deberia permitir:
\begin{itemize}
	\item Las mismas acciones que estando conectado directamente con la placa de adquisicion, solamente que via interfaz grafica, y no de comando.
	\item Enviar cualquier comando que pueda ser interpretado por la placa de adquisicion
	\item Establecer intervalos de tiempo usando hora y fecha en los que se debe medir sobre cierto canal
\end{itemize}
\item Deberia guardar datos de mediciones e informacion sobre transacciones en general entre ambas placas en una base de datos local, accesible via la interfaz grafica 
\end{itemize}


% section requerimientos_de_la_iteracion (end)

\section{Desarrollo} % (fold)
\label{sec:desarrollo}

% section desarrollo (end)

\section{Pruebas} % (fold)
\label{sec:pruebas}

% section pruebas (end)

\section{Resultados} % (fold)
\label{sec:resultados}

% section resultados (end)

% chapter iteracion_6 (end)