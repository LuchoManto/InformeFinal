\chapter{Introducción} % (fold)
\label{cha:introduccion}

\section{Motivacion} % (fold)
\label{sec:motivacion}

% section motivacion (end)

\section{Objetivos} % (fold)
\label{sec:objetivos}






% section objetivos (end)

\section{Herramientas de modelado} % (fold)
\label{sec:herramientas_de_modelado}

Para facilitar la documentación, decidimos utilizar SysUML para documentar nuestro sistema. Para esto, utilizamos la herramienta Enterprise Architect, que nos permite partir de un modelo base y reutilizar los componentes en todos los diagramas, generando vínculos que facilitan el entendimiento general del sistema.

% section herramientas_de_modelado (end)

\section{Requerimientos} % (fold)
\label{sec:requerimientos}

\subsection{Requerimientos funcionales} % (fold)
\label{sub:requerimientos_funcionales}

\begin{itemize}
	\item Se debería poder recibir datos de sensores analógicos en modo singular o equilibrado
	\item Se debería poder contar eventos de señales digitales cuadradas
	\item Se debería poder configurar una ganancia especifica y un tiempo entre mediciones especifico para cada entrada
	\item Los parámetros que corresponden a la ganancia, el modo de conversión, el tiempo entre conversiones, y la cantidad de canales habilitados para la conversión y el conteo de eventos, deberían ser configurables a través de una interfaz de linea de comando
	\item Se debería usar un protocolo serial para la interfaz de linea de comando entre el sistema y el usuario
	\item El usuario debería poder configurar el sistema conectándolo a un ordenador u otro sistema que soporte comunicación serial.
	\item Se debería poder enviar datos de mediciones y posibles meta-datos a través de un canal de comunicación mediante un protocolo serial.
\end{itemize}

% subsection requerimientos_funcionales (end)

\subsection{Requerimientos no funcionales} % (fold)
\label{sub:requerimientos_no_funcionales}

\begin{itemize}
	\item Debería consumir lo menos posible
\end{itemize}

% subsection requerimientos_no_funcionales (end)

% section requerimientos (end)

\section{Características generales} % (fold)
\label{sec:caracteristicas_generales}

En una primera aproximación del sistema a construir, se puede establecer que se trata de un sistema capaz de recibir señales de distintos sensores digitales y analógicos. En caso que sean analógicos, convertir las señales a digital usando un conversor A-D de alta ganancia, que permita tanto entradas singulares como equilibradas. Luego de ser convertidas, estas señales deben ser enviadas mediante un protocolo serial a otro sistema que realice las acciones que deba realizar en función de los datos enviados.

Se puede establecer que en nuestro contexto interactúan 3 actores: Un usuario, uno o mas sensores, y un controlador que por el momento no hacemos mención de sus características, por lo que lo llamamos simplemente ``controlador''.

\begin{figure}[h]
  \centering
  \includegraphics[width=0.80\textwidth, height = 9cm]{contexto1}
  \caption{Contexto del sistema}\label{fig:contexto1}
\end{figure}

% section caracteristicas_generales (end)

\section{Casos de uso} % (fold)
\label{sec:casos_de_uso}

Dado el contexto, el sistema se puede describir de una manera general mediante un diagrama de caso de uso. El diagrama puede verse en la figura \ref{fig:casouso1}. En esta figura, se pueden ver los requerimientos plasmados en los casos de uso.

\begin{figure}[h]
  \centering
  \includegraphics[width=0.80\textwidth, height = 11cm]{casouso1}
  \caption{Diagrama de caso de uso del sistema de adquisición}\label{fig:casouso1}
\end{figure}

% section casos_de_uso (end)

% chapter introduccion (end)