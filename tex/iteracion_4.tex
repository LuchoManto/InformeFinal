\chapter{Iteracion 4: Diseño final de hardware} % (fold)
\label{cha:iteracion_4}

\section{Introduccion} % (fold)
\label{sec:introduccion}

Nos propusimos poder duplicar la cantidad de recursos de la placa, ya que el principal problema era medir varios sensores de modo diferencial, con 8 entradas solo podriamos tener 4 sensores como maximo. Entonces decidimos poner dos microcontroladores a la placa. Pasar de 8 a 16 entradas analogicas y utilizar todas las entradas digitales. 
Para no agrandar el tamaño decidimos rediseñar la placa, y como primera medida la diseñamos para poder construirla doble capa.
En esta iteracion mostraremos como se realizo el diseño y la construccion de esta nueva placa.

% section introduccion (end)

\section{Requerimientos de la iteracion} % (fold)
\label{sec:requerimientos_de_la_iteracion}

\begin{itemize}
	\item La placa debe tener dos microcontroladores C8051f352.
	\item La placa debe tener 16 entradas analogicas.
	\item La placa debe tener 32 entradas digitales.
	\item La placa debe tener 2 salidas seriales. 
\end{itemize}

% section requerimientos_de_la_iteracion (end)

\section{Desarrollo} % (fold)
\label{sec:desarrollo}

\subsection{Diseño Esquematico}
\label{diseño_esquematico2}

Para simplificar la explicación del diagrama, lo que haremos en esta sección es dividir el circuito entero en subcircuitos mas simples.

\subsubsection{Entradas Analogicas}
\label{entradas_analogicas2}

Las entradas analogicas con sus filtros se mantuvieron exactamente igual que en el diseño de la primera placa. Para cada una de las entradas se le colocaria un filtro pasa-bajo RC como se muestra en la figura \ref{fig:esquematicoFiltro}.

% subsubsection entradas_analogicas2 (end)

\subsubsection{Circuito Salida Serial}
\label{salida_serial2}

La idea de tener una placa a la que se conecten varios sensores analógicos y digitales es que se pueda colocar el lugares remotos, por lo que decidimos que la salida serial deje de ser RS-232 y colocarle dos salidas seriales nivel TTL con un RX y un TX para cada microcontrolador, ademas colocando éste tipo de salida se ahorra mucho espacio, así que pudimos reducir el tamaño de la placa. 

% subsubsection salida_serial2 (end)

% subsection diseño_esquematico2 (end)

% section desarrollo (end)

\section{Pruebas} % (fold)
\label{sec:pruebas}

% section pruebas (end)

\section{Resultados} % (fold)
\label{sec:resultados}

% section resultados (end)

% chapter iteracion_4 (end)