\chapter{Iteracion 3: Primer prototipo de Hardware} % (fold)
\label{cha:iteracion_3}


\section{Introduccion} % (fold)
\label{sec:introduccion}

la primera placa que fue una verga. tenia el rs232, tenia 8 entradas, tenia la alimentacion separada de la entrada para la programacion del micro. osea podia alimentarse mediante el debugger o alimentacion externa. se adapto el diseño de la placa para poder programar el micro con el debugger de silicon labs. hay que tener en cuenta que nos basamos en el diseño hecho por silicon labs de la placa de desarrollo c8051f352 que teniamos en el lac. la vamos a haber construido en esta iteracion, y lo unico que llego a hacer fue conectarse con la ide. nada mas, el resto no anduvo nada. 

% section introduccion (end)

\section{Requerimientos de la iteracion} % (fold)
\label{sec:requerimientos_de_la_iteracion}

Diseñar y construir un prototipo de placa con las siguientes caracteristicas

\begin{itemize}
\item El circuito debe incluir en su diseño aquellos requisitos de hardware impuestos por el mismo microcontrolador(?)
\item Al circuito se le deben poder conectar 8 entradas analogicas.
\item Al circuito se le deben poder conectar 4 entradas de eventos digitales externos.
\item Las entradas analogicas deben tener filtros para mejorar la inmunidad al ruido.
\item Se debe incluir en el diseño el circuito necesario para soportar comunicacion via RS232
\item La placa deberia poder alimentarse a traves de una fuente de tension externa.
\item Se deberia poder conectar el debugger del microcontrolador a la placa para poder programarlo.
\item El circuito de programacion del microcontrolador deberia estar separado la placa principal.
\end{itemize}


% section requerimientos_de_la_iteracion (end)

\section{Desarrollo} % (fold)
\label{sec:desarrollo}

% section desarrollo (end)

\section{Pruebas} % (fold)
\label{sec:pruebas}

% section pruebas (end)

\section{Resultados} % (fold)
\label{sec:resultados}

% section resultados (end)

% chapter iteracion_3 (end)