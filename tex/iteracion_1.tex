\chapter{Iteración 1: Investigación y selección de recursos} % (fold)
\label{cha:iteracion_1}

\section{Introducción} % (fold)
\label{it1:sec:introduccion}

La selección del microcontrolador es el primer paso a superar en la investigación para la construcción de la plataforma. Los requerimientos y objetivos definidos, dieron pie a la conformación de una tabla comparativa donde colocamos distintos microcontroladores que parecían reunir las características necesarias para ser potencialmente elegidos para nuestra plataforma. En esta iteración, explicamos aquellos parámetros analizados para cada controlador, y el proceso de análisis para la selección final.

% en esta iteracion de lo que hay que hablar es desde el momento en el que no teniamos nada al momento en que dijimos "tenemos esto para desde aca arrancar. " . osea, teniamos que el mejor micro para trabajar era el 8051 y teniamos la placa de desarrollo asi que arrancamos a huevear ahi. lo primero que hicimos fue probar que todas las cosas andaran bien, hicimos programas de prueba para todas las funcionalidades, que eran: el adc, el UART, el SMBus (que no anduvo nunca), y los contadores de eventos.

% despues de las prubas arrancamos a diseñar el software. pero eso creo que ya va para la iteracion 2. en esta iteracion va entonces, toda la parte de investigacion, podemos poner alguna que otra salida de prueba de las funcionalidades probadas.. q se yo alguna imagen cosas asi. frut

% section introduccion (end)

\section{Requerimientos de la iteración} % (fold)
\label{it1:sec:requerimientos_de_la_iteracion}

Seleccionar un microcontrolador que reúna las siguientes características:
\begin{itemize}
  \item Debería poder convertir a digital 8 o más señales analógicas. (derivado de 1.1)
  \item Debería poder elegir entre conversión de canal único o diferencial. (derivado de 1.2)
  \item Debería tener contadores para contar eventos de hasta 4 fuentes distintas. (derivado de 1.3)
  \item Debería tener ganancia configurable para el conversor analógico digital. (derivado de 1.4)
  \item Debería transmitir los datos digitales a través de un protocolo serial a alguna otra placa o procesador. (derivado de 1.6)
\end{itemize}


% section requerimientos_de_la_iteracion (end)

\section{Desarrollo} % (fold)
\label{it1:sec:desarrollo}


\subsubsection{Selección de parámetros clave} % (fold)
\label{ssub:seleccion_de_parametros_clave}

Cada microcontrolador reúne una serie de características que hace que sean distintos uno de otro. Para nuestro sistema, solo son algunos los parámetros que deben tenerse en cuenta para una selección acorde a nuestros requerimientos. Estos parámetros fueron seleccionados según el análisis de requerimientos, distinguiendo aquellos de mayor riesgo y los de menor riesgo. 

\begin{itemize}
  \item \textbf{RAM:} No es un requisito principal, pero en caso de tener que decidir entre dos microcontroladores similares, el tamaño de la memoria puede ser un factor para tomar la decisión final.
  \item \textbf{Cantidad de canales del ADC:} Mientras mas canales se tengan, mas señales analógicas de entrada pueden haber, y mas señales de sensores se podrán procesar simultáneamente
  \item \textbf{Referencia:} Nos dice si los pines del ADC se pueden usar como entrada diferencial o únicamente con referencia a GND. Esto es porque en el caso que haya 16 pines para el ADC y puedan usarse todos como entrada diferencial, se podrán usar como máximo la mitad de los pines, es decir 8.
  \item \textbf{Resolución:} Es la cantidad de bits con la que se representa el dato convertido. A mayor resolución, mayor precisión de la conversión.
  \item \textbf{Ganancia:} Una buena ganancia interna en el micro es necesaria para una amplificación de la señal, ya que reduce la relación señal-ruido. Este parámetro es clave si se quiere trabajar con sensores que funcionan a voltajes muy pequeños en ambientes susceptibles al ruido eléctrico.
  \item \textbf{Contadores:} Cantidad de contadores en el microcontrolador que se utilizarían como contadores de eventos(es necesario que la fuente que incrementa el contador provenga de eventos externos y no interiores al microcontrolador).
  \item \textbf{Modos de bajo consumo:} Si tiene mas de un modo de bajo consumo, es mas simple lograr que el sistema se encuentre el mayor tiempo posible consumiendo lo menor posible.
  \item \textbf{Puerto serie:} Interfaces seriales que soporta el micro. De manera mínima se necesita que soporten $I^{2}$C y UART.
  \item \textbf{Dimensión:} Dimensión del micro. El tamaño de la placa debería ser lo menor posible por lo que mientras mas pequeño el micro, mejor.
  \item \textbf{Cantidad de pines:} Dependiendo del encapsulado, habrá una cantidad de pines. La cantidad puede afectar el tamaño y la complejidad de la placa.
  \item \textbf{Precio:} Costo del integrado.
\end{itemize}

% subsection parametros_tenidos_en_cuenta_en_la_seleccion_del_microcontrolador (end)

\subsection{Análisis de microcontroladores disponibles en el mercado} % (fold)
\label{it1:sub:analisis_de_microcontroladores}

Teniendo en cuenta los parámetros mas importantes realizamos una búsqueda en la web, de distintos microcontroladores potenciales para nuestra plataforma. Cada microcontrolador nuevo encontrado con potencial de selección era analizado por separado teniendo en cuenta los parámetros descriptos en la sección \ref{ssub:seleccion_de_parametros_clave}. La tabla \ref{tabla_micros} muestra los microcontroladores considerados.

% Table generated by Excel2LaTeX from sheet 'Sheet1'
\clearpage
\begin{landscape} % TABLA DE MICROS

\begin{table}[!]
\centering
\begin{flushleft}
% Table generated by Excel2LaTeX from sheet 'Sheet1'
\scalebox{0.68}{
\begin{tabular}{|c|c|c|c|c|c|c|c|c|c|c|c|c|}
\hline
Fabricante & Modelo & RAM(K) & canales ADC & Referencia & Resolución & ganancia & Contadores & low power & puerto serie & Dimensión (') &   Pins &   Us\$ \\
\hline
 Intel & 8XC51GB &    256 &      8 &    GND & 8 bits &     no & 3 (16 bits) &     si & salida y entrada RS232 &      N &     32 &      N \\
\hline
Silicon Labs & C8051F352 &    768 &      8 & DIF/GND & 24 bits &   128x & 4 (16 bits) &     si & Smbus/$I^{2}$C, UART, SPI & 0,35x0,35 &     32 &    2,3 \\
\hline
 Atmel & AT89C5115 &    256 &      8 &    GND & 8/10 bits &     no & 3 (16 bits) &     si & UART (3 modos Full Duplex) & 0,34x0,34 &     32 &     10 \\
\hline
Microchip & PIC18F4550 &     32 &     13 &    GND & 10 bits &     no & 4(8 y 16 bits) &     si & SPI, $I^{2}$C, UART/USART, USB & 0,47x0,47 &     44 &   5,36 \\
\hline
   Nec & PD78C17 &   1024 &      8 &    GND & 8 bits &     no & 2 (8 bits) &     si & Msbus/$I^{2}$C & 0,92x0,70 &     64 &      N \\
\hline
 Maxim & DS4830 & 1024x16 &     16 & DIF/GND & 13 bits &     no & 2(16 bits) &     si & SPI, $I^{2}$C & 0,2x0,2 &     40 &    7,5 \\
\hline
   NXP & LPC1110 &      4 &      8 &    GND & 10 bits &     no & 2(32 bits) &     si & $I^{2}$C, UART, Soporte RS-485 & 0,42x0,51 &     20 &    2,5 \\
\hline
 Atmel & ATSAM3A8C &    256 &     16 & DIF/GND & 12 bits &     no & 9(32 bits) &     si & USB, SPI & 0,63x0,63 &     63 &    2,4 \\
\hline
 Atmel & ATSAM3S1A &     64 &      8 & DIF/GND & 10/12 bits & 1x,2x,4x & 3(16 bits) &     si & USB, $I^{2}$C, SPI & 0,35x0,35 &     44 &    2,5 \\
\hline
 Atmel & ATSAM3S1C &     16 &     16 & DIF/GND & 10/12 bits & 1x,2x,4x & 6(16 bits) &     si & USB, $I^{2}$C, SPI & 0,63x0,63 &     74 &    2,5 \\
\hline
 Atmel & ATSAMD21J &    256 &     20 & DIF/GND & 12 bits &    16x & 5 (16 bits) &     si & 1 USB 2.0 + 6 $I^{2}$C/USART/SPI & 0,47x0,47 &     64 &      3 \\
\hline
 Atmel & ATSAMD21G &    256 &     14 & DIF/GND & 12 bits &    16x & 3 (16 bits) &     si & 1 USB 2.0 + 6 $I^{2}$C/USART/SPI & 0,35x0,35 &     48 &    2,5 \\
\hline
 Atmel & ATSAMD21E &    256 &     10 & DIF/GND & 12 bits &    16x & 3 (16 bits) &     si & 1 USB 2.0 + 4 $I^{2}$C/USART/SPI & 0,35x0,35 &     32 &    2,5 \\
\hline
Texas Instr & MSP430F5340 &     64 &      9 &    GND & 12 bits &     2x & 7 (distintas) &     si & SPI, $I^{2}$C, UART & 0,3x0,3 &     48 &    3,3 \\
\hline
    ST & STM32F373CX &    256 &      4 &    GND & 12, 16 bits &    32x & 17 (distintas) &     si & 2 $I^{2}$C, 3 SIP, 3 USART, 1 USB & 0,35x0,35 &     48 &    2,5 \\
\hline
Atmel AVR & ATmega128 &    128 & 8 (2 c/gain) & 7 DIF, 8 GND & 10 bits & 1x, 10x, 200x & 4 (8 y 16) &     si & USART, SPI & 0,6x0,6 &     64 &      8 \\
\hline
\end{tabular}



}
\end{flushleft}
  \caption{Microcontroladores considerados}\label{tabla_micros}
\end{table}

\end{landscape}

% section analisis_de_microcontroladores (end)


% section desarrollo (end)

\section{Selección} % (fold)
\label{it1:sec:seleccion}

Decidimos optar por el microcontrolador Silicon Labs C8051F352. Haciendo un análisis de la tabla \ref{tabla_micros}, llegamos a la conclusión de que este microcontrolador es el mejor se adapta a los parámetros mas importantes, como ser la cantidad de entradas analógicas para la conversión, la resolución, la ganancia programable, los contadores de eventos, y el protocolo serial SMBus (compatible con $I^{2}$C).

Para el desarrollo de distintas pruebas sobre el desempeño del microcontrolador, obtuvimos una placa de desarrollo Silicon Labs C8051F352. Una ilustración de esta placa de desarrollo puede verse en la Figura \ref{fig:placadedesarrollosiliconlabs}

\begin{figure}[h]
  \centering
  \includegraphics[width=0.50\textwidth, height = 4cm]{placadedesarrollosiliconlabs.jpg}
  \caption{}\label{fig:placadedesarrollosiliconlabs}
\end{figure}


% section selección (end)   

% chapter iteracion_1 (end)