\chapter{Iteración 2: Primer prototipo de software} % (fold)
\label{cha:iteracion_2}

\section{Introducción} % (fold)
\label{it2:sec:introduccion}

% fue cuando empezamos a tirar fruta. habiendo elegido el microcontrolador empezamos a probar todas las funcionalidades: El adc, los contadores de eventos, el modulo serial. Despues arrancamos a diseñar el primer prototipo. Consideramos que segun los requerimientos tenia que se un sistema que ofrezca algun tipo de interfaz para que un usuario interactue con el, para que pueda configurarle los parametros segun lo que se quiere lograr.

Los requerimientos de mayor riesgo, son calificados de esta manera porque, de no poder cumplirse, logran que la plataforma sea obsoleta con respecto a sus objetivos principales. La manera de verificar que estos requerimientos son posibles de cumplir, es seleccionar un microcontrolador y luego desarrollar un programa de prueba que compruebe las distintas funcionalidades ligadas a dichos requerimientos. En la actual iteración, verificamos la selección, realizando un programa de configuracion básica para la plataforma.

\subsection{Objetivo de la iteración} % (fold)
\label{sub:objetivo_de_la_iteracion}

El objetivo principal de esta iteración, es conformar un primer prototipo de software. Este prototipo no debe necesariamente cumplir con todos los requerimientos, pero si debe verificar el funcionamiento de los módulos del C8051F352 ligados a aquellos requerimientos de mayor riesgo: El conversor analógico-digital; la ganancia programable; los contadores; y el módulo serial.

% subsection objetivo_de_la_iteracion (end)


% section introduccion (end)

\section{Requerimientos de la iteración} % (fold)
\label{it2:sec:requerimientos_de_la_iteracion}

Teniendo en cuenta los requerimientos princípiales y el microcontrolador elegido, definimos los siguientes requerimientos para el software de la plataforma:

\begin{itemize}
\item Se debería poder utilizar el conversor ADC del microcontrolador para transformar señales analógicas de fuentes externas a datos digitales, tanto en modo canal único como canal diferencial. (derivado de 1.1)
\item Se debería poder utilizar los contadores del microcontrolador para contar eventos de fuentes externas. (derivado de 1.3)
\item Se debería poder utilizar el modulo serial UART y SMBus del microcontrolador para enviar datos a otra placa o microprocesador. (derivado de 1.6)
\item Para cada canal del conversor:
\begin{itemize}
\item Se debería poder habilitar y deshabilitar el canal para la medición. (derivado de 1.2)
\item Se debería poder configurar el modo de medición (canal único o diferencial). En caso de ser canal único debería especificarse un solo canal, y dos canales para modo diferencial. (derivado de 1.2)
\item Se debería poder configurar un tiempo de intervalo entre cada medición. (derivado de 1.5)
\end{itemize}
\item Para cada contador:
\begin{itemize}
\item Se debería poder contar eventos de hasta 4 fuentes externas, utilizando los contadores del microcontrolador. (derivado de 1.3)
\end{itemize}
\item Datos de mediciones o meta-datos deberían ser enviados por comunicación serial utilizando protocolo SMBus o UART, según elija el usuario. (derivado de 1.6)

\end{itemize}


% section requerimientos_de_la_iteracion (end)

\section{Desarrollo} % (fold)
\label{it2:sec:desarrollo}

En esta sección se describirán las distintas partes del software y la interacción entre ellas para cumplir con los requerimientos. Esta descripción se hará mediante modelos del patrón de diseño SysML. El programa en si no fue diseñado mediante este patrón, pero realizamos un proceso de ingeniería inversa para facilitar la compresión del lector en el presente informe.

\subsection{Modelos estáticos} % (fold)
\label{it2:sub:modelos_estaticos}

El diagrama de caso de uso de la Figura \ref{fig:casouso1}, establece de manera básica aquello que se quiere lograr con el programa embebido en la plataforma. En cada caso de uso, pueden visualizarse las distintas acciones que el programa debería realizar. Como el programa esta desarrollado en lenguaje C, no es posible conformar un diagrama de clases. En lugar de esto, se separo en distintos módulos que agrupan funciones de características similares. Estos módulos están ilustrados como bloques en la Figura \ref{it2:fig:bloquesprimeraiteracionsoftware}. Cada bloque representa distintas partes principales del programa. Se pueden ver los nombres de cada función y el tipo de retorno en cada uno de ellos. Con esto ultimo, se intenta dar una idea de las acciones realizadas por las funciones de cada modulo. %Una descripcion mas detallada esta en la documentacion del programa \ref{documentacionsoftware}.%

El caso de uso ``configurar'' esta generalizado. Las acciones que incluye este caso son:
\begin{itemize}
  \item Configurar la interfaz serial
  \item Configurar canal en modo singular
  \item Configurar canal en modo equilibrado
  \item Configurar contador de eventos
  \item Configurar ganancia del conversor
  \item Configurar intervalo de medición para conversión analógica
\end{itemize}


\begin{figure}[h]
  \centering
  \includegraphics[width=1\textwidth, height = 11.5cm]{bloquesprimeraiteracionsoftware}
  \caption{Diagrama de bloques de la primera iteracion de software}\label{it2:fig:bloquesprimeraiteracionsoftware}
\end{figure}

El objetivo de los diagramas es lograr una descripción gráfica del sistema. Es necesario destacar que, desde el principio, la evolución del programa ocasiono que los diseños de los módulos fueran cambiando. Los cambios fueron debidos a múltiples razones: particularidades del funcionamiento del microcontrolador que no se tuvieron en cuenta, limitaciones del entorno, cambios en los requerimientos, etcétera. 

A continuación, elaboramos una pequeña descripción de los bloques que aparecen en el diagrama de la Figura \ref{it2:fig:bloquesprimeraiteracionsoftware}.

\begin{itemize}
  \item \textbf{Main o Bloque Principal}: Orquestador del resto de los módulos. Tiene, además, la rutina principal donde se entra al modo de configuración, para luego iniciar el modo de conversiones continuas.
  \item \textbf{Contador:} Inicializa el hardware de los contadores de eventos del microcontrolador, y contiene las funciones para obtener los valores de los mismos.
  \item \textbf{Interfaz de Usuario:} Levanta la interfaz con la que interactúa el usuario para establecer los parámetros configurables del sistema, además de iniciar el modo de conversiones continuas.
  \item \textbf{Configurador:} Accede a los registros de configuración de los distintos módulos del microcontrolador. Contiene funciones que realizan todas las configuraciones necesarias para poder hacer funcionar cada modulo del programa.
  \item \textbf{Serial:} Configura el modulo serial para el envió y recepción de caracteres. Puede ser UART o $I^{2}$C.
\end{itemize}

% subsection modelos_estaticos (end)

Los fabricantes del microcontrolador proporcionan una librería en C para trabajar con los registros del procesador. Todos los bloques excepto el de la interfaz gráfica, manipulan registros para realizar las distintas acciones que le corresponden según la función que se ejecuta.  


\subsection{Modelos dinámicos} % (fold)
\label{it2:sub:modelos_dinamicos}

Utilizamos los modelos dinámicos del patrón SysML para describir las distintas interacciones entre los distintos módulos del programa. En esta sección, describimos las interacciones entre el usuario y el sistema que cubren los requerimientos principales. Las funciones involucradas en cada interacción son las mismas declaradas en el diagrama de bloques del sistema. Los distintos flujos que se describen en estos modelos fueron desarrollados iterativamente, es decir, son producto de un desarrollo incremental, hasta llegar al estado que se describe en esta iteración. \\

Un flujo de ejecución típico del programa consiste en una configuración y un inicio de conversiones continuas. Las conversiones continuas son un loop infinito donde se convierte cíclicamente sobre los canales configurados para la conversión y se envían los datos convertidos por interfaz serial, utilizando el mismo canal serial por donde se realiza el intercambio de comandos para la configuración. Cuando se interrumpen las conversiones continuas, el programa devuelve al usuario la posibilidad de configurar el sistema, dando fin a las conversiones y el envió de datos. \\

La Figura \ref{fig:secuenciaconfiguracionbasica} muestra un diagrama de secuencia para una configuración de un canal del conversor en modo canal único. Realizada la configuración, la Figura \ref{fig:secuenciaconversioncontinua} muestra como el sistema se establece en modo de conversión continua, entrando en un loop infinito de conversión y envió de telemetría.

\begin{figure}[h]
  \centering
  \includegraphics[width=1\textwidth, height = 9cm]{secuenciaconfiguracionbasica}
  \caption{Diagrama de secuencia para una configuración del conversor. Establece el canal 3 en modo canal único}\label{fig:secuenciaconfiguracionbasica}
\end{figure}

\begin{figure}[h]
  \centering
  \includegraphics[width=1\textwidth, height = 9cm]{secuenciaconversioncontinua}
  \caption{Diagrama de secuencia para la activación de conversiones en modo continuo luego de una configuración previa.}\label{fig:secuenciaconversioncontinua}
\end{figure}



\subsection{El modulo conversor} % (fold)
\label{it2:sub:el_modulo_conversor}

Los registros del microcontrolador permiten manipular el conversor de la siguiente manera (para propositos de la logica de conversion):

\begin{itemize}
  \item Se puede establecer de 1 hasta 8 pines en modo canal unico
  \item Se puede establecer de 1 hasta 4 pares de pines en modo diferencial
  \item Se puede establecer un nivel de ganancia de x2 a x128 para todos los canales
\end{itemize}

Esta lista nos permite concluir que las posibilidades que permite el microcontrolador son limitadas para el uso que se le quiere dar. El modulo conversor del programa debería, por un lado, permitir la configuración necesaria de los registros del conversor; y además, debería tener funciones que permitan una configuración mas compleja de los canales, teniendo en cuenta los requerimientos principales.
Con esto en mente, se desarrollaron distintas funciones que realizan configuraciones según ciertos parámetros de entrada, ingresados por el usuario. Para facilitar la configuración, se creo un buffer denominado ``buffer de canales''. Este buffer contiene 8 posiciones, y cada una de ellas se corresponde con uno de los canales del conversor. Se elaborara sobre este buffer mas adelante.

Estas son las funciones del modulo conversor:

\begin{itemize}
  \item \textit{convertir}: Interactúa con el hardware del microcontrolador para obtener la ultima conversion realizada. Es ejecutada únicamente por la rutina de interrupción del ADC (iniciada en cada " End of Conversion")
  \item \textit{cargar\_buffer\_single}: Según un parámetro de entrada entero entre 0 y 7, se carga una posición dentro del buffer de canales con el numero "1", indicando que el canal que se corresponde con esa posición en el buffer se debe leer en modo canal único.
  \item \textit{cargar\_buffer\_dif}: Según un parámetro de entrada entero y par entre 0 y 6, se carga una posición dentro del buffer de canales con el numero "2", indicando que el canal ligado a esa posición en el buffer, y el siguiente en numero, deben ser leídos en modo diferencial.
  \item \textit{cambiar\_pin}: Cambia el canal por donde se medirá en la próxima conversión.
\end{itemize}
% subsection modelos_dinamicos (end)

\subsection{Buffer de canales y lógica de cambio de canal en modo de conversiones continuas} % (fold)
\label{it2:sub:buffer_de_canales_y_logica_de_cambio_de_canal_en_modo_de_conversiones_continuas}

Las funciones \textit{cargar\_buffer\_single} y \textit{cargar\_buffer\_dif}, establecen el modo (canal único o diferencial) en el que se leerá un canal. Las 8 posiciones del buffer de canales representan, cada una, un canal distinto del ADC. Los valores posibles para estas posiciones son 0, 1 y 2; estos valores representan, respectivamente, que el canal esta deshabilitado, en modo canal único, o en modo diferencial. Estas funciones, al ser ejecutadas, cargan algún numero en el buffer según que función es la que se ejecuto. El buffer se inicializa por defecto en 0, por lo que inicialmente ningún canal esta habilitado para convertir. \\

Cuando se inicia la conversión continua, se convierte continuamente en un loop infinito. Al final de cada conversión, se llama a la función \textit{cambiar\_pin}. Esta función utiliza el buffer de canales para obtener cual es el proximo canal a medir. Esto lo hace manejando un índice que comienza en 0, se incrementa hasta 7 y vuelve a 0; recorriendo cíclicamente el buffer de canales. 
En cada ejecución, analiza una posición buffer segun el valor del índice. El valor de la posición del buffer dice si se debe medir en modo canal único o diferencial, o si es necesario saltear el pin. Una descripción gráfica del comportamiento de esta función puede verse en el diagrama de actividades de la Figura \ref{fig:actividadescambiarpinit2} \\


\begin{figure}[h]
  \centering
  \includegraphics[width=1\textwidth, height = 8cm]{actividadescambiarpinit2}
  \caption{Diagrama de actividad de la funcion cambiar\_pin}\label{fig:actividadescambiarpinit2}
\end{figure}

%Este algoritmo fue el primer prototipo de logica de asignacion y cambio de canal para la lectura de mediciones del conversor. A simple vista es posible ver que los potenciales problemas son muchos. Para empezar, si se asigna el pin 5 en modo canal unico y el modo 4 en modo diferencial, el programa intentara medir en dos modos distintos el mismo canal, dando seguramente resultados inconsistentes. 

% subsection buffer_de_conversion_y_logica_de_cambio_de_canal_en_conversion_continua (end)

% subsection logica_de_las_funciones_del_conversor (end)

\subsection{Lógica de las funciones de la interfaz} % (fold)
\label{it2:sub:logica_de_las_funciones_de_la_interfaz}

% Aca es necesario partir el tema en 2.. porque al principio empezamos haciendo todo mal. quisimos hacder una especioe de menu interactivo que le diera la posivilidad al usuario de poder elegir entre distintas opciones de este menu, desplegadas en forma de arbol, para que pudiera elegir la configuracion que quisiera. O sea, arrancaba al principio con un menu general donde ibas navegando hasta poner la configuracion que querias. Asi empezamos, y seguimos con esa logica hasta que se hizo insufrible el programa. Cada vez que queriamos agregar una nueva opcion eran muchas lineas de codigo para cambiar. ahi es cuando investigando me llego el concepto de lo que se llama MMLo "man machine language". Es una logica muy simple. es como un paradigma de interfaz hombre maquina, donde el usuario lo que hace es enviar unos comandos estandarizados y una serie de argumentos, armados en base a una expersion regular. Cisco y Unix Bash usan este paradigma para interactuar con los usuarios. la idea es tan simple como poderosa, lo unico que habia que hacer era decidir como iba a ser el formato de entrada. El analisis posterior es una secuencia que parsea la entrada en busca del comando y los parametros. con el comando, se sabe que funcion ejecutar, y los argumentos son pasados como parametros para esta funcion. Este paradigma es mas complicado de sacar andando, pero es mucho mas escalable que el del menu. a la hora de agregar funcionalidades nuevas se hacia en mucho menos tiempo. Ademas, hace que el programa sea mucho mas facil de testear con unit testing.

La interfaz de usuario tuvo dos implementaciones, siendo la segunda un reemplazo de la primera.

La primera interfaz fue de tipo ``Menu Based Interface'' o interfaz basada en menú, donde las acciones a realizar son ofrecidas por la interfaz y seleccionadas por un usuario. Cada acción esta clasificada según el grupo de acciones al que pertenezca. Por ejemplo: para configurar el pin 4 del ADC en modo canal único, es necesario navegar por las opciones del menú de la forma: 

\begin{equation}
\noindent
\text{Configuración} \rightarrow \text{Configurar Pin ADC} \rightarrow \text{Modo} \rightarrow \text{Único} \rightarrow \text{Canal} \rightarrow \text{4} 
\end{equation}



Siguiendo este patrón de diseño, en cada adición de una nueva acción, era necesario buscar el grupo al que pertenecía y programar la lógica necesaria para ejecutar la acción en base a la navegación del usuario dentro del menú. Esta forma de desarrollo probo ser muy poco escalable, y fue necesario cambiarla por la dificultad que implicaba a la hora de agregar nuevas funcionalidades. \\

La segunda interfaz diseñada fue de tipo ``Command line interface''. En este tipo de interfaz, el usuario ingresa un comando y uno o varios argumentos en un formato que respeta una expresión regular. Bajo este concepto, se necesita un parser o analizador de comandos que extraiga de la entrada el comando y los argumentos, y realice acciones según ellos. En nuestro caso, cada comando representa una función distinta de nuestro programa, y los argumentos pasan a ser parámetros de la función. \\

Luego de implementar un analizador de comandos, agregar funcionalidades al programa fue mas simple y rápido con una interfaz basada en linea de comando que con una basada en menú. Por lo que el primer diseño fue descartado y el segundo se mantuvo. \\

\begin{figure}[h]
  \centering
  \includegraphics[width=0.40\textwidth, height = 12cm]{actividadinterfaz}
  \caption[Diagrama de actividad del comportamiento de la interfaz de usuario]{Diagrama de actividad del comportamiento de la interfaz de usuario. Se toma la entrada y se analiza para obtener el comando y los argumentos. Según el comando, se ejecuta una u otra función dentro del programa. Los parámetros necesarios para ejecutar la función se incluyen como argumentos dentro de un arreglo que esta en la estructura global. Este arreglo es luego utilizado por la función llamada para obtener los argumentos ingresados junto con el comando.}\label{fig:actividadinterfaz}
\end{figure}

\subsubsection{Formato de los comandos} % (fold)
\label{it2:ssub:formato_de_los_comandos}

Todos los comandos que pueden ingresarse por la interfaz respetan una expresión regular. En el caso nuestro, es la siguiente:

\begin{equation}
\wedge \backslash w \{2,3\} (,\backslash d )\{0,2\} \$
\end{equation}

Dicho de otra manera, el comando ingresado esta conformado por 2 caracteres alfabéticos como mínimo y 3 como máximo. El comando puede o no tener argumentos; si los tiene, son dos argumentos numéricos como máximo, separados por comas. El comando y los argumentos también están separados por una coma.

% subsubsection formato_de_los_comandos (end)

% subsection logica_de_las_funciones_de_la_interfaz (end)

\subsection{Contadores de eventos} % (fold)
\label{it2:sub:contadores_de_eventos}


El modulo contador contiene las funciones involucradas en el proceso de obtener las cuentas de los contadores de eventos activos en el microcontrolador. Para obtener el valor de una cuenta, simplemente se lee el registro ligado al contador que lleva dicha cuenta. Las funciones de este modulo obtienen la cuenta y la devuelven en forma de cadena de caracteres. \\

Los requerimientos principales indican que es necesario poder contar eventos de 3 o 4 fuentes distintas. Al momento de tomar la decisión de cual microcontrolador utilizar, tuvimos en cuenta que este microcontrolador contaba con 4 contadores que podían ser configurados para que tomen fuentes externas para realizar el conteo. En pleno desarrollo del programa, llegamos a la conclusión de que esto no iba a ser posible por las siguientes razones:

\begin{itemize}
  \item El modulo serial del microcontrolador hace uso de uno de los contadores para el generador de baudios.
  \item Los contadores 2 y 3 comparten la fuente externa de donde ingresan los eventos que se cuentan.
\end{itemize}

Teniendo en cuenta esto, solo es posible contar eventos de 2 fuentes en simultaneo, utilizando timer 0 y timer 2 o timer 3. \\

La reducción de recursos disponibles para cumplir los requerimientos no fueron suficientes para que sea practico un cambio de microcontrolador, siendo que el resto de las funcionalidades funcionaban correctamente y como era esperado (Con excepción de SMBus, que se desarrolla mas adelante). Para solucionar este problema, se considero la posibilidad de un diseño de una plataforma con dos microcontroladores, duplicando la cantidad de recursos, y llegando asi al requisito de 4 contadores de eventos de fuentes externas.

% subsection contadores_de_eventos (end)

\subsection{El modulo serial} % (fold)
\label{it2:sub:el_modulo_serial}

Para hacer funcionar el modulo serial, es necesario utilizar uno de los timers como generador de baudios. La implementación de la función que configura el modulo serial del microcontrolador vino ya implementada por Silicon Labs como ejemplo de programa. De todas formas, este ejemplo funcionaba correctamente en nuestro programa y no fue necesario hacerle ningún cambio, mas que modificaciones en el baud rate. \\

Según los requerimientos, era necesario poder alternar entre SMBus (equivalente a $I^{2}$C) y UART para la interfaz serial. Sin embargo, a pesar de haber realizado múltiples pruebas, no fue posible lograr que funcione el modulo SMBus. Luego de acordarlo con el director, decidimos utilizar UART como único protocolo para la interfaz serial.

% subsection el_modulo_serial (end)

% section desarrollo (end)

\section{Pruebas} % (fold)
\label{it2:sec:pruebas}


\subsection{Tests unitarios} % (fold)
\label{it2:sub:tests_unitarios}

Cada una de las funciones, tanto si interactúan con el hardware del micro como si no, están testeadas utilizando unit-testing en C, con ayuda del framework ``minunit''\cite{minunit}. Los unit-tests se ejecutaban de la misma manera que se ejecutaba el programa principal, y corría todos los tests, dando como resultado la cantidad de tests que pasaron y los que fallaron.

\begin{table}[h]
\caption{Tests unitarios}
\label{it2:tab:tests_unitarios}
\begin{tabular}{p{2cm} p{7cm} p{2cm}}
\hline
\cellcolor[HTML]{68CBD0}Nombre & \cellcolor[HTML]{68CBD0}Descripción & \cellcolor[HTML]{68CBD0}Estado  \\ \hline
Modo único & Intenta configurar un pin en modo canal único & {\color[HTML]{009901} Pass}  \\ \hline
Modo diferencial & intenta configurar un pin en modo diferencial & {\color[HTML]{009901} Pass} \\ \hline
GDI & Intenta ingresar el comando GDI con combinaciones correctas e incorrectas de parámetros, esperando que no haya error en los primeros casos y que si los haya en los segundos & {\color[HTML]{009901} Pass} \\ \hline
GSE & Análogo a test\_GDI & {\color[HTML]{009901} Pass}  \\ \hline
SSE & Análogo a test\_GDI & {\color[HTML]{009901} Pass}  \\ \hline
SDI & Análogo a test\_GDI & {\color[HTML]{009901} Pass}  \\ \hline
Obtener entrada & Pide al usuario que ingrese una serie de comandos y verifica que la respuesta del programa sea la esperada & {\color[HTML]{009901} Pass}   \\ \hline
Tests iniciales & Comprueba el funcionamiento del modulo serial del microcontrolador y el conversor analógico digital & {\color[HTML]{009901} Pass}   \\ \hline
Parpadear led & Prueba el funcionamiento de la plataforma con un programa que hace parpadear un led. Este test esta pensado para ser ejecutado en la plataforma a construir & {\color[HTML]{CB0000} Fail}   \\
\end{tabular}
\end{table}

% subsection tests_unitarios (end)

\subsection{Tests de sistema} % (fold)
\label{it2:sub:tests_de_sistema}

Las pruebas de sistema se realizaron utilizando la placa de desarrollo mencionada en la sección \ref{it1:sec:seleccion}. Al no tener todavía una implementación hecha del sistema, se utilizo esta placa para testear el código embebido en el microcontrolador, que en definitiva seria el mismo que iría en la placa a construir

\begin{table}[h]
\caption{Test de sistema 1: Comportamiento esperado del conversor en modo canal unico}
\label{it2:tab:testsistema1}
\begin{tabular}{p{2cm} p{9cm}}
\multicolumn{2}{c}{\cellcolor[HTML]{68CBD0}{\color[HTML]{000000} Prueba de sistema}} \\
Prueba \#        & 1 \\
\hline
Nombre           & Comportamiento esperado del conversor en modo canal único   \\     
\hline
Requerimiento      & 1.1   \\
\hline
Descripción      & Se conecta un generador de tensión en uno de los canales del conversor y se mide en ese canal en modo canal único, y de esta forma comprobar no tan solo que se este midiendo, sino además que estas mediciones estén calibradas para este modo. \\             
\hline
Pre-condiciones  & \tabitem Sistema configurado con un pin en modo canal único \\
                 & \tabitem Generador de tensión conectado al pin configurado  \\
                 & \tabitem Computadora conectada al sistema mediante cable serial RS-232 \\
                 & \tabitem Lector de canal serial abierto en la computadora  \\
                 & \tabitem Sistema midiendo en modo de conversiones continuas \\
\hline

Post-condiciones & Los datos de las conversiones que aparezcan en el programa lector de interfaz serial en la computadora deberían corresponderse con los valores de tensión provenientes del generador                     
\\
\hline
Secuencia  & \tabitem Establecimos una tensión de 1,5 voltios en el generador \\
           & \tabitem La medición resultante en el lector de canal serial era la misma que la del generador  \\
           & \tabitem Variamos la tensión en el generador y la variación se correspondía en la lectura \\
           & \tabitem Cambiamos a tensiones negativas y verificamos que la lectura daba 0  \\
           & \tabitem Cambiamos a tensiones por sobre la tensión de referencia (por defecto 2,5V), y verificamos que el sensor no da valores superiores a 2,5V \\
\hline
Resultados       & Las mediciones dieron resultados coherentes, con algunas variaciones esperadas debido al ruido presente en el ambiente.
\end{tabular}
\end{table}

\begin{table}[h]
\caption{Test de sistema 2: Comportamiento esperado del conversor en modo diferencial}
\label{it2:tab:testsistema2}
\begin{tabular}{p{2cm} p{9cm}}
\multicolumn{2}{c}{\cellcolor[HTML]{68CBD0}{\color[HTML]{000000} Prueba de sistema}} \\
Prueba \#        & 2 \\
\hline
Nombre           & Comportamiento esperado del conversor en modo diferencial \\
\hline
Requerimiento     & 1.1   \\
\hline
Descripción      & Se conecta un generador de tensión en uno de los canales del conversor y se mide en ese canal en modo diferencial, y de esta forma comprobar no tan solo que se este midiendo, sino además que estas mediciones estén calibradas para este modo. Además, se comprueba el funcionamiento de las mediciones en modo diferencial, intentando medir tensiones negativas.                                                                                  \\
\hline
Pre-condiciones  & \tabitem Sistema configurado con un pin en modo diferencial \\
                 & \tabitem Generador de tensión conectado al par de pines configurados en modo diferencial. Con el borne positivo en uno y masa en el otro\\
                 & \tabitem Computadora conectada al sistema mediante cable serial RS-232 \\
                 & \tabitem Lector de canal serial abierto en la computadora \\
                 & \tabitem Sistema midiendo en modo de conversiones continuas\\
\hline

Post-condiciones & Los datos de las conversiones que aparezcan en el lector de interfaz serial en la computadora deberían corresponderse con los valores de tensión provenientes del generador  
\\ 
\hline
Secuencia  & \tabitem Establecimos una tensión de 1 voltio en el generador \\
           & \tabitem La medición resultante en el lector de canal serial era distinta de la del generador. Leía 2V\\
           & \tabitem Variamos la tensión en el generador y la variación se correspondía en la lectura \\
           & \tabitem Subimos la tensión a 1.25V y la lectura dio 2.5V, a partir de ahí, por mas que se subía la tensión en el generador la lectura no daba mas de 2.5V. \\
           & \tabitem Cambiamos a tensiones negativas y la lectura no era la esperada pero era consistente. Cuando en el generador había un 0, leía 1.25V, a medida que la tensión se hacia mas negativa, decrementaba aproximadamente al mismo paso que el generador, hasta que el generador llegaba a -1.5V, donde leía un 0  \\
\hline
Resultados       & Las mediciones no dieron los resultados esperados, aunque en si los resultados eran consistentes: los cambios en el generador se reproducían en la medición, aunque los datos no eran los mismos. Esto se debe a que cuando se mide en modo diferencial, el cero pasa a ser la tensión de referencia dividida en dos, para representar los valores negativos. La lógica del programa no tenia en cuenta esto, y media las tensiones de forma desfasada.
\end{tabular}
\end{table}

\begin{table}[h]
\centering
\caption{Test de sistema 3: Comportamiento esperado del contador de eventos}
\label{it2:tab:testsistema3}
\begin{tabular}{p{2cm} p{9cm}}
\multicolumn{2}{c}{\cellcolor[HTML]{68CBD0}{\color[HTML]{000000} Prueba de sistema}} \\
Prueba \#        & 3 \\
\hline
Nombre           & Comportamiento esperado del contador de eventos \\
\hline
Requerimiento & 1.3 \\
\hline
Descripción      & Se conecta un generador de onda cuadrada en una entrada GPIO que pueda servir como entrada a algún contador de eventos. Si el contador esta activo, se debería poder ver que el numero de cuentas incrementa al ritmo de la frecuencia configurada en el generador \\
\hline
Pre-condiciones  & \tabitem Sistema con Timer0 configurado como un contador de eventos activo \\
                 & \tabitem Generador de frecuencia conectado al pin configurado como contador. \\
                 & \tabitem Computadora conectada al sistema mediante cable serial RS-232 \\
                 & \tabitem Lector de canal serial abierto en la computadora \\
                 & \tabitem Lectura continua mediante el canal serial del valor de la cuenta de Timer0\\
\hline

Post-condiciones & El contador debería contar con una frecuencia igual a la del generador de onda cuadrada
\\ 
\hline
Secuencia  & \tabitem Establecimos una frecuencia de 1000 Hz en el generador \\
           & \tabitem La salida del lector de canal serial mostraba que el contador incrementaba 1000 veces por segundo \\
           & \tabitem Se espero por algunos minutos, verificando que la cuenta sea consistente\\
           & \tabitem Se disminuyo el valor de frecuencia a 100 Hz y se repitió la misma verificación. \\
           & \tabitem Se disminuyo el valor de frecuencia a 0 Hz y se verifico que el valor de la cuenta quede fijo sin modificarse por 5 minutos\\
\hline
Resultados       & La cuenta era consistente con la frecuencia del generador. El contador de eventos funcionaba correctamente.

\end{tabular}
\end{table}
% subsection tests_de_sistema (end)

% section pruebas (end)
\section{Conclusiones de la iteración 2} % (fold)
\label{it2:sec:conclusiones_de_la_iteracion_2}

Basándonos en el desarrollo y las pruebas de la iteración, listamos los avances teniendo en cuenta los requerimientos principales y los particulares de esta iteración:

\begin{itemize}
\item El programa utiliza el conversor del microcontrolador para transformar señales analógicas de fuentes externas a datos digitales. (1.1)
\item El programa utiliza los contadores del microcontrolador para contar eventos de fuentes externas. (1.1)
\item El programa utiliza el modulo serial UART del microcontrolador para enviar los datos a otra placa o microprocesador (1.1)
\item No es posible utilizar el modulo SMBus del microcontrolador. (1.6)
\item Los datos de conversión correspondientes al numero de pin y el modo de conversión no son enviados junto con la medición. 
\item Para cada canal del conversor:
\begin{itemize}
\item El usuario puede habilitar o inhabilitar el canal para la medición (1.1)
\item El usuario puede configurar el modo de medición (canal único o diferencial). En caso de ser canal único especificarse un solo canal, y dos canales para modo diferencial. (1.1)
\item No es posible configurar un tiempo de intervalo entre cada medición (1.5)
\end{itemize}
\item Para cada contador:
\begin{itemize}
\item El usuario puede habilitar o inhabilitar el conteo de eventos. (1.3)
\end{itemize}
\item No es posible elegir el protocolo serial para comunicarse con la placa o microprocesador externo que recibirá los datos, obligadamente debe usarse UART. (1.6)

\end{itemize}

\subsection{Objetivos para la próxima iteración} % (fold)
\label{it2:sub:objetivos_para_la_proxima_iteracion}

Teniendo un prototipo de software, lo ideal fue comenzar a pensar en un diseño de hardware en donde hacerlo funcionar. Por lo que el objetivo de la iteración 3 fue realizar una implementación de la placa de instrumentación en un PCB; y una vez construido, realizar las pruebas que aseguren el funcionamiento del mismo.


% subsection objetivos_para_la_proxima_iteracion (end)
% section conclusiones_de_la_iteracion_2 (end)
% chapter iteracion_2 (end)