\chapter{Iteracion 0: Orden de las iteraciones} % (fold)
\label{cha:iteracion_0}

\section{Introduccion} % (fold)
\label{sec:introduccion}

El objetivo de esta iteracion es explicar el porque del orden elegido para el desarrollo de cada iteracion. Este orden proviene directamente de la dependencia de los requerimientos entre unos y otros. En caso de tener que elegir entre requerimientos independientes entre si, optamos por el de mayor riesgo.  \\

Los riesgos de los requerimientos del proyecto estan estimados en la tabla PONER RIESGO DE REQUERIMIENTOS

% queremos indicar el orden de las iteraciones. no necesariamente tienen cualquier orden. Las iteraciones vienen directo de los requerimientos. Los requerimientos dependen a veces unos de otros, y eso determina el orden. Cuando no dependen uno de otro, se decide cual hacer primero segun el riesgo del requerimiento. Ponele, que el sistema cuente eventos y convierta de analogico a digital es de riesgo alto porque es clave. No se puede hacer nada si no se decide primero que micro usar,, y asi y asi.

% section introduccion (end)

\section{Determinacion del orden} % (fold)
\label{sec:determinacion_del_orden}

Como se puede ver en la tabla PONER RIESGO DE REQUERIMIENTOS, los requerimientos de mayor riesgo son la conversion, el conteo de eventos, y la ganancia. La razon de esto fue simplemente que el objetivo principal puesto por nuestro director era el de tener una placa generica para la conversion analogico digital y el conteo de eventos, el resto era secundario. \\

Aunque estos requerimientos sean los de mayor riesgo, dependen de otro. Fue necesaria la seleccion de un microcontrolador que reuna las caracteristicas necesarias para cumplir con los requerimientos de mayor riesgo y la mayoria del resto. Es por esto que la primera iteracion fue dedicada a la seleccion de un microcontrolador. \\

Teniendo un microcontrolador elegido, decidimos comprobar que al menos lo escencial funcionara correctamente. Esto es: Comprobar que funcione el conversor, la ganancia programable, y el conteo de eventos. La segunda iteracion fue dedicada a la experimentacion con el microcontrolador elegido con su placa de desarrollo, y la conformacion de un primer prototipo de programa que cumpla con los requerimientos de mayor riesgos. En el desarrollo de esta iteracion y las pruebas de sistema, surgieron cuestiones que provocaron cambios en los requerimientos, pero que no comprometieron la eleccion del microcontrolador.  \\

Llegados a esta instancia, teniamos un prototipo de software que testeaba que los requerimientos de mayor riesgo estaban cubiertos con el microcontrolador elegido. El proximo paso era comprobar que el diseño de la placa de desarrollo podia ser implementado por nosotros, en una placa que reuna los requisitos de nuestro sistema, haciendo que pase de ser de un proposito general a un proposito especifico. En la tercera iteracion, se realizo un prototipo de placa de instrumentacion con el diseño de la placa de desarrollo elegida. Las mismas pruebas de la segunda iteracion se realizaron junto con las pruebas de la tercera iteracion, quedando asi un prototipo de sistema que cumplia con los requerimientos de mayor riesgo. \\

Al final de la tercera iteracion, en lo que respecta unicamente a la placa de instrumentacion, teniamos un sistema que cumplia con los requerimientos minimos requeridos por nuestro director. Lo siguiente a realizar es pasar del prototipado a un sistema final. Las iteraciones 4 y 5 fueron destinadas a finalizar las implementaciones de hardware y sofware. \En la iteracion 4 se volvio a realizar el hardware, rediseñando ciertos aspectos y duplicando los recursos debido a los cambios en los requerimientos, y en la iteracion 5 se continuo con el desarrollo del software, dandole ya las funcionalidades suficientes para cubrir los requerimientos vinculados al programa. \\

A este punto, teniamos ya lo principal del proyecto: Una placa de instrumentacion implementada en una PCB, con 16 entradas analogicas con ganancia programable, y 4 contadores de eventos. El resto de las iteraciones fueron casos de prueba y mejoras al sistema. 
La iteracion 6 fue dedicada al desarrollo de un programa dentro de una placa Raspberry Pi que gestione la placa de instrumentacion hecha.  \\

En la ultima iteracion, el objetivo principal era poner a prueba el sistema con un sensor de campo electrostatico, y ademas acondicionar el sistema para con este sensor, teniendo un prototipo de producto, con la idea de hacerlo comerciable. Por cuestiones de tiempos y otras circunstancias, quedo como un prototipo para unicamente fines academicos. La iteracion 7 cubre el desarrollo de la adaptacion del sistema realizado hasta el momento a este sensor. \\

% section determinacion_del_orden (end)

% poner a prube el sistema y ademas usar las funcionalidades del microcontrolador para el funcionamiento del sensor

% cuales son los primeros requerimientos que hay que cumplir?

% -hay que convertir señales de analogico a digital
% -hay que amplificar las señales convertidas
% -hay que contar eventos

% con eso solo tenes lo primero y principal. pero antes que eso, no hay nada mas?

% lo primero que hay que hacer es elegir el microcontrolador a usar, porque eso nos determina el funcionamiento del resto. Entonces la primera iteracion va a ser eso.

% Lo segundo que hay que hacer es asegurarse de que con el micro elegido se puedan hacer las tres cosas que se mencionaron recien, entonces la segunda iteracion se trata de experimentar y conformar un programa dentro del micro que haga lo basico y sacar conclusiones en base al desarrollo y las pruebas

% Una vez hecho esto ya se pueden hacer dos cosas: se puede seguir trabajando sobre el programa para tenerlo mas completo y que cumpla con todos los requerimientos principales mas los qeu hayan surgido en el desarrollo, o se puede arrancar con la implementacion del hardware de la placa de adquisicion. Pero ambos se pueden hacer en paralelo porque el software se puede probar en la placa de desarrollo. Es mas, es mejor tener el software ya bastante avanzado asi se puede probar mas facil que la placa nueva ande para lo que queremos que ande y no para otra cosa. Por el estado en que estaba el programa ya probaba lo escencial, asi que decidimos que la tercera iteracion iba a ser el desarrollo del hardware. Esta enconces fue la iteracion 3

% Terminado un prototipo de hardware con pruebas hechas, hicimos una iteracion mas. Las cosas que implementaba el programa eran suficientes para probar que la placa armada ande bien, asi que no se le dio bola por una iteracion mas y se siguio trabajando en el harware. Esta fue la iteracion 4

% Teniendo el hardware ya diseñado e implementado en una placa que ande, seguimos con el software, ya terminandolo y dejandolo listo para el uso en la placa nueva. Esta es la iteracion 5

% con eso ya estaba la placa de adquisicion que era lo principal del proyecto. Despues surgieron los anexos. El primero que surgio fue el del sensor y despues el de la raspberry. Ambos se pueden hacer en paralelo, pero siendo el sensor un caso de prueba, se deja para el final. Primero se desarrolla un prototipo de gestionador de dispositivos IOT para tener algo ya mas completo y configurable y bonito. esta entonces es la iteracion 6

% La 7ma y ultima iteracion es el caso de prueba con el sensor. Al ser un caso de prueba, va al ultimo


% chapter iteracion_0 (end)